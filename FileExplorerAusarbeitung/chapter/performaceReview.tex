\section{Performanceoptimierung}\label{sec:Performanceoptimierung}

Performance-Optimierung ist ein zentraler Bestandteil der Softwareentwicklung, insbesondere bei Anwendungen, die regelmäßig mit
großen Datenmengen oder komplexen Verarbeitungsprozessen arbeiten. Ziel ist es, Ressourcen wie CPU-Zeit, Arbeitsspeicher und I/O
effizient zu nutzen, um die Reaktionsgeschwindigkeit und Gesamtleistung der Anwendung zu verbessern. Eine performante Anwendung
trägt nicht nur zu einem besseren Nutzererlebnis bei, sondern reduziert auch Energieverbrauch und Hardwareanforderungen.

Ein nützliches Werkzeug zur Performance-Analyse sind sogenannte \textit{Flamegraphs}. Diese visualisieren die Aufrufstruktur eines
Programms und zeigen an, wie viel Zeit in welchen Funktionen verbracht wird. Die Breite eines Balkens in einem Flamegraph steht
dabei für die Häufigkeit bzw. Dauer der Ausführung einer Funktion innerhalb eines Profiling-Zeitraums – je breiter der Balken,
desto höher der Ressourcenverbrauch an dieser Stelle.

Flamegraphs helfen, sogenannte Hotspots im Code zu identifizieren – also Funktionen oder Abschnitte, die besonders viel
Rechenzeit beanspruchen. Durch gezielte Analyse dieser Bereiche können Engpässe erkannt und durch algorithmische Verbesserungen,
Caching oder Parallelisierung gezielt behoben werden. Besonders hilfreich ist dabei die hierarchische Darstellung: Sie zeigt,
welche Funktionen von welchen anderen aufgerufen wurden, was eine Ursachenanalyse erheblich erleichtert.
