\section{Performanceoptimierung}\label{sec:Performanceoptimierung}

Performance-Optimierung stellt einen essenziellen Bestandteil Softwareentwicklung dar, besonders bei Anwendungen, die
häufig mit großen Datenvolumina oder komplexen Berechnungen arbeiten müssen. Ziel einer solchen Optimierung ist die effiziente
Nutzung von Ressourcen wie CPU-Leistung, Arbeitsspeicher und Eingabe-/Ausgabesystemen (I/O), um die Reaktionszeiten und die
allgemeine Performance einer Applikation maßgeblich zu verbessern. Neben einer Steigerung des Nutzerkomforts tragen performante
Anwendungen auch zu einer Verringerung des Energieverbrauchs und zu niedrigeren Anforderungen an die Hardware bei.

Ein besonders wirkungsvolles Hilfsmittel zur Analyse und Optimierung der Performance sind sogenannte \textit{Flamegraphs}. Diese
Diagramme visualisieren detailliert die Aufrufhierarchie eines Programms und veranschaulichen, wie lange und wie oft bestimmte
Funktionen ausgeführt wurden. Im Flamegraph wird die Dauer der Funktionsausführung durch die Breite eines Balkens repräsentiert.
Je breiter der Balken, desto größer ist der Anteil der Rechenzeit, die diese Funktion beansprucht hat.

Mithilfe von Flamegraphs lassen sich sogenannte Hotspots im Programmcode präzise identifizieren. Hotspots sind jene Funktionen
oder Codeabschnitte, die überproportional viel Rechenleistung verbrauchen. Durch eine gezielte Untersuchung dieser kritischen
Bereiche lassen sich Leistungsengpässe aufdecken und mittels gezielter algorithmischer Verbesserungen, effizientem Caching oder
Parallelisierung beheben. Wir haben diese verwendet um während der Entwicklung Methoden zu überprüfen, welche mit größeren
Datenmengen hantieren. Hierbei war es uns wichtig, dass wir gerade, da es um das Filesystem geht, so performant wie möglich
hantieren. An vielen Stellen ist weiterhin Raum für Verbesserungen durch verbesserte Speicherverwaltung. Um sich ein Bild zu
machen, wie diese aufgebaut sind, befinden sich im Anhang Beispiele (\ref{fig:first_flame_graph} \ref{fig:second_flame_graph}) für
die von uns erstellten Graphen. Erstellt haben wir diese mit einem der vielen Flamegraph Tools \cite{flamegraph_rs}. Wir haben uns
für dieses entschieden, da es direkt über cargo erreichbar ist und reibungslos mit Rust funktioniert. 

(Die dargestellten Graphen zeigen Spitzen, welche durch die rekursive Suche in den Unterverzeichnissen entstehen. Da diese
Verzeichnisse von Testläufen generiert wurden, ist es wichtig, zwischen Setup-Code und dem eigentlichen Anwendungscode zu
differenzieren. Die gezeigten Graphen dienen ausschließlich der Veranschaulichung typischer Merkmale und sollten daher nicht als
reale Performance-Metriken interpretiert werden.)
