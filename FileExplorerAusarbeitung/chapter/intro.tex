\section{Einleitung}

Die Entwicklung einer eigenen Software, die genau jene Funktionen besitzt, welche man sich im Alltag
immer gewünscht hat, bietet spannende Möglichkeiten – sowohl kreativ als auch praktisch. Gerade im
Bereich der Dateiverwaltung, die jeder täglich nutzt, fällt oft auf, dass Standard-File-Explorer
häufig nur das Nötigste bieten: Dateien anzeigen und einfache Bedienoberflächen bereitstellen. Dabei
bleiben viele Anforderungen, die moderne Nutzer wirklich haben, meist unbeachtet.

Genau deshalb war es für uns besonders interessant, selbst einen File Explorer zu entwickeln, der
unseren eigenen Vorstellungen entspricht. Dieses Projekt gab uns nicht nur die Freiheit, lange
gewünschte Funktionen einzubauen, sondern ermöglichte auch, ganz praktisch auszuprobieren, welche
Ansätze tatsächlich hilfreich sind: Wie gestaltet man eine Benutzeroberfläche, die Menschen wirklich
unterstützt, entlastet und informativ ist? Welche Strukturen und Automatismen sorgen im Alltag für
echte Effizienz?

Die Idee zu diesem Projekt entstand durch ein YouTube-Video von Conaticus \cite{conaticus01}, in
dem er seinen Ansatz für einen schnellen File Explorer vorstellt. Da wir persönlichen Kontakt mit
ihm aufnehmen konnten und er von unserer Beteiligung begeistert war, entschieden wir uns gezielt für
dieses Projekt. Schon beim ersten Blick auf das Konzept hatten wir konkrete Vorstellungen für
Erweiterungen, da auch wir selbst bisher viele Funktionen bei existierenden Explorern vermisst
hatten. Außerdem schildert Conaticus in einem weiteren Video ausführlich die Herausforderungen rund
um die Suchfunktion \cite{conaticus02}, was wir als interessante Aufgabe ansahen.

Technisch basiert unser Projekt auf dem Framework Tauri \cite{tauri2025}, das eine klare Trennung
der Anwendung ermöglicht: Einerseits gibt es ein Frontend, das mit beliebten Web-Technologien
entwickelt wird, andererseits die Verarbeitungslogik, die in Rust implementiert ist. Zum Zeitpunkt
unserer Übernahme war das Frontend noch nicht vollständig fertiggestellt, da der ursprüngliche Fokus
vor allem auf Funktionalität lag und weniger auf Design und Nutzerfreundlichkeit. Im Austausch mit
Conaticus bestätigte er uns, dass er ebenfalls Interesse an einer attraktiven Benutzeroberfläche
habe. Aus diesem Grund haben wir uns entschieden, das Frontend mit React neu und ansprechend zu
gestalten.
