\section{Einleitung}

Die Entwicklung einer eigenen Software mit genau jenen Funktionen, welche man sich im Alltag
immer gewünscht hat, bietet sowohl kreativ als auch praktisch spannende Möglichkeiten. Insbesondere
im Bereich der Dateiverwaltung, die täglich genutzt wird, bieten Standard-File-Explorer, wie der
Windows-Explorer oder auf macOS der Finder, meist nur das nötigste: Dateien anzeigen und einfache
Bedienoberflächen bereitstellen. Dabei bleiben viele Anforderungen des modernen Nutzers meist
unbeachtet.

Insbesondere deshalb war es für uns interessant einen File Explorer nach unseren eigenen
Vorstellungen zu entwickeln. Dieses Projekt gab uns nicht nur die Freiheit, lange
gewünschte Funktionen einzubauen, sondern ermöglichte auch auszuprobieren, welche
Ansätze tatsächlich hilfreich sind: Wie gestaltet man eine Benutzeroberfläche, die Menschen wirklich
unterstützt, entlastet und informativ ist? Welche Strukturen und Automatismen sorgen im Alltag für
echte Effizienz?

Die Idee zu diesem Projekt entstand durch ein YouTube-Video von Conaticus \cite{conaticus01}, in
dem er seinen Ansatz für einen schnellen File Explorer vorstellt. Da wir persönlichen Kontakt mit
ihm aufnehmen konnten und er von unserer Beteiligung begeistert war, entschieden wir uns für
dieses Projekt. Schon beim ersten Blick auf das Konzept hatten wir konkrete Vorstellungen für
Erweiterungen, da auch wir selbst bisher viele Funktionen bei existierenden Explorern vermisst
hatten. Außerdem schildert Conaticus in einem weiteren Video ausführlich die Herausforderungen rund
um die Suchfunktion \cite{conaticus02}, was wir ebenfalls als interessante Aufgabe ansahen.

Technisch basiert unser Projekt auf dem Framework Tauri \cite{tauri2025}, das eine klare Trennung
der Anwendung ermöglicht: Einerseits gibt es ein Frontend, welches mit React entwickelt wurde und
andererseits die Verarbeitungslogik, die in Rust implementiert ist. Zum Zeitpunkt unserer
Projektstartes war das Frontend noch nicht vollständig fertiggestellt, da der ursprüngliche Fokus
vor allem auf Funktionalität und weniger auf Design und Nutzerfreundlichkeit lag. Im Austausch mit
Conaticus bestätigte er uns ebenfalls Interesse an einer attraktiven Benutzeroberfläche. Aus diesem
Grund haben wir uns entschieden, das Frontend mit React neu und ansprechend zu gestalten.