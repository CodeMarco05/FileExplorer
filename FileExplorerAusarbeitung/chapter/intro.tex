\section{Einleitung}

Die Entwicklung von Software für den eigenen Gebrauch – ausgestattet mit exakt jenen Funktionen, die man sich über Jahre hinweg im
Alltag gewünscht hat – stellt nicht nur eine kreative, sondern auch eine methodisch interessante Herangehensweise dar.
Insbesondere im Bereich der Dateiverwaltung, der tagtäglich durch den Einsatz verschiedenster File Explorer geprägt ist, zeigt
sich ein eklatantes Missverhältnis zwischen praktischer Nutzung und funktionaler Erfüllung. Bestehende Systeme sind oftmals primär
darauf ausgerichtet, Dateien darzustellen und ein minimales Interface bereitzustellen; tiefergehende Anforderungen moderner
Nutzerinnen und Nutzer bleiben hingegen häufig unbeachtet.

Die Möglichkeit, einen File Explorer vollständig neu zu denken, eröffnet daher nicht nur den Freiraum zur Integration lang
ersehnter Funktionen, sondern stellt zugleich ein erkenntnisreiches Experiment im Bereich der Mensch-Computer-Interaktion dar.
Welche Strukturen, Navigationsprinzipien und Automatisierungsansätze begünstigen tatsächliche Effizienz? Wie lässt sich ein
Interface gestalten, das nicht nur funktioniert, sondern die Nutzerin bzw. den Nutzer aktiv unterstützt, informiert und entlastet?
Eine solche Neuentwicklung ist nicht bloß als technischer Fortschritt zu verstehen, sondern als reflektierter Beitrag zur
Weiterentwicklung alltäglicher Softwarewerkzeuge.

Auf das zugrunde liegende Projekt wurden wir durch ein YouTube-Video aufmerksam \cite{connaticus01}, in dem der Autor seine Idee
eines besonders schnellen File Explorers erläutert.

Wir haben dieses Projekt gezielt ausgewählt, da wir mit dem Ersteller persönlich Kontakt aufgenommen haben und dieser sich über
unser Interesse erfreut zeigte. Bereits bei der ersten Auseinandersetzung mit dem Konzept kamen unserem Team konkrete
Erweiterungsideen, da auch wir bestimmte Funktionen in einem Datei-Explorer vermissen. Zudem erläutert der Autor in einem weiteren
Video detaillierte Schwierigkeiten im Zusammenhang mit der Suchfunktion \cite{connaticus02}, die wir als spannende Herausforderung
empfanden.

Das Projekt basiert auf dem Framework Tauri \cite{tauri2025}, das es ermöglicht, Anwendungen in zwei Module zu unterteilen:
einerseits das User Interface, welches mit den gängigsten Frontend-Frameworks entwickelt werden kann \cite{tauri_frontends},
andererseits die Verarbeitungsschicht, die in Rust \cite{rust} implementiert wird. Bei der Übernahme des Projekts war das Frontend
noch nicht vollständig umgesetzt, da der ursprüngliche Fokus auf der Funktionalität lag und weniger auf einer ansprechenden
Gestaltung. In der Rücksprache mit dem Projektinitiator bestätigte dieser, dass ein ästhetisch überzeugenderes Frontend auch in
seinem Interesse liege. Aus diesem Grund entschieden wir uns für die Verwendung von React zur Gestaltung der Benutzeroberfläche.

Durch den Einsatz von Rust für die Verarbeitungskomponenten profitieren wir von den bekannten Vorteilen dieser Programmiersprache:
hoher Laufzeitgeschwindigkeit und effiziente Ressourcennutzung – vorausgesetzt, die Programmierung erfolgt entsprechend
sorgfältig.
