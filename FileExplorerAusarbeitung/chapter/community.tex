\section{Projektplanung und Kommunikation}\label{sec:Absprache mit Ersteller und Community} 

Gerade bei umfangreicheren Projekten ist es entscheidend, eine klare und strukturierte Organisation
sicherzustellen. Dazu gehört nicht nur eine interne Absprache im Team, sondern auch die
Kommunikation mit externen Beteiligten, um frühzeitig mögliche Verbesserungen zu identifizieren und
gemeinsam umzusetzen um von der Ursprünglichen Vision nicht abzuweichen. 

\subsection{Zusammenarbeit mit Community und dem Ersteller}\label{sec:Zusammenarbeit mit Community und dem Ersteller} % (fold)

Um unser Projekt von Anfang an eng an die ursprüngliche Vision von Conaticus anzulehnen und
gleichzeitig sicherzustellen, dass unsere Ideen und Umsetzungen in seinem Sinne sind, haben wir
frühzeitig den direkten Kontakt zu ihm aufgenommen. Hierfür sind wir seinem Discord-Server
beigetreten, auf dem ein Austausch mit seiner Community stattfand. Durch diesen Dialog hatten wir
die Möglichkeit, konkrete Fragen zu stellen und konnten von Conaticus selbst explizit die Erlaubnis
und Zustimmung einholen, Änderungen und Erweiterungen an seinem bestehenden Projekt vorzunehmen,
welche auch in seinem Entwicklungssinne sind. Diese Genehmigung wurde uns schriftlich erteilt und
kann jederzeit auf Nachfrage eingesehen werden. Zusätzlich haben wir auch die Pull Requests auf
GitHub sowie die Kommentarsektionen unter seinen YouTube-Videos ausführlich betrachtet und mögliche
Features durchgesprochen. Dadurch konnten wir erkennen, welche Funktionen besonders gefragt sind,
und den Programmcode so strukturieren, dass zukünftige Features einfacher integriert werden können.
Etwa zur Halbzeit des Projektes führten wir ein Online-Meeting mit Conaticus persönlich durch, um
direktes Feedback zu unseren bisherigen Umsetzungen einzuholen. Das Gespräch verlief ausgesprochen
positiv und Conaticus war bezüglich des neuen Interfaces und der Funktionen sehr erfreut.

\subsection{Kommunikation im Team}

Zu Beginn des Projekts haben wir überlegt, wie wir unsere Zusammenarbeit möglichst klar und
strukturiert gestalten können. Dabei war uns wichtig, den Code sinnvoll aufzubauen und gleichzeitig
den Überblick über Aufgaben und Fortschritte zu behalten. Zur Organisation haben wir Trello genutzt:
Dort haben wir das Projekt in einzelne Aufgaben aufgeteilt, diese den Teammitgliedern zugewiesen und
den Fortschritt über Listen wie „To Do“, „In Bearbeitung“ und „Erledigt“ festgehalten. So konnten
wir effizient zusammenarbeiten und jederzeit nachvollziehen, was bereits umgesetzt wurde und was
noch ansteht.
